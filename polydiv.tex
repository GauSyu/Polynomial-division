%LaTeX Template 
%%	by	Xu Gao (gausyu@gmail.com)
%%LPPL: http://www.latex-project.org/lppl.txt
%------------------------
%The following is the Preamble
%------------------------
\documentclass[11pt]{article}
%This is the document class. 
%%	Most common: article, beamer, book, etc.
%%	11pt means the default font size is 11pt. One can also use 10pt, 11pt, or 12pt.
%%	See https://en.wikibooks.org/wiki/LaTeX/Document_Structure#Document_classes for explaination.
%------------------------
%Packages
%------------------------
\usepackage[a4paper,total={6in, 9in}]{geometry} 
%This aims to customize page layout
\usepackage[T1]{fontenc}
%Using T1 font
\usepackage[leqno]{mathtools}
%The main MATH package (for whom may wonder, mathtools load amsmath automatically, so no need \usepackage{amsmath})
\usepackage{amsthm}
%This defines the theorem enviroments
\usepackage{amssymb}
%Math symbols
\usepackage{bm}
%Bold math fonts, provide \bm{} command
\usepackage[scr=rsfs,cal=euler]{mathalpha}
%Math fonts 
%%The package provides means of loading maths alphabets (such as are normally addressed via macros \mathcal, \mathbb, \mathfrak and \mathscr)
%%How to use? "scr=" set what font \mathscr use, "cal=" set what font \mathcal use, "bb=" set what font \mathbb use, and "frak=" set what font \mathfrak use.
\usepackage{tikz}
%Drawing
\usepackage{graphicx}
%Support for graphics
\usepackage{hyperref}
\usepackage[nameinlink]{cleveref}
%Hyper links
\hypersetup{
	colorlinks=true,
	linkcolor=blue,
	urlcolor=magenta
}
%Color links
\usepackage{etoolbox}
%Programming
\usepackage[shortlabels]{enumitem}
%Enable enumerate modification
\usepackage{tcolorbox}
%provides an environment for coloured and framed text boxes with a heading line.
\tcbset{colback=white}
%------------------------
%Theorem-like Enviroments
%------------------------
\crefname{equation}{}{}
\theoremstyle{plain}
%This is the theorem style, plain means blod title and italic body
\newtheorem{theorem}{Theorem}
%This is how you define a theorem-like enviroment.
\newtheorem{lemma}[equation]{Lemma}
\newtheorem{proposition}[equation]{Proposition}
%The option [equation] means the lemmas are numbered following the same system of equation.
\newtheorem*{propstar}{Proposition}
%Star version defines unnumbered enviroments
\theoremstyle{definition}
%This is the theorem style, definition means blod title and normal body
\newtheorem*{defn}{Definition}
\newtheorem{example}[equation]{Example}
%The option [equation] means the examples are numbered following the same system of equation.
\newtheorem{problem}{Problem}
%This is the enviroment used as problems in HWs.
\theoremstyle{remark}
%This is the theorem style, remark means italic title and normal body
\newtheorem*{remark}{Remark}
\newtheorem*{hint}{Hint}
\newtheorem*{optional}{Optional}
%Remark enviroment
\numberwithin{equation}{problem}
%Let equations be numbered with in each problems.
\NewDocumentEnvironment{solution} {o} 
{\begin{proof}[\IfNoValueTF{#1}{\textbf{Solution}}{\textbf{Solution} (#1)}]}{\end{proof}}
%This defines the enviroment for you to write solutions.
%------------------------
\newlist{listinprob}{enumerate}{1}
\setlist[listinprob]{label=(\alph{listinprobi}),
                  ref=\theproblem.(\alph{listinprobi})}
\crefname{listinprobi}{problem}{problems}
\Crefname{listinprobi}{Problem}{Problems}
%This defines a new list type used in the enviroment problem.
%------------------------
%DocumentCommands
%------------------------
%Notations for number sets
\NewDocumentCommand \N {} { \mathbb{N} }%Natural numbers
\NewDocumentCommand \Z {} { \mathbb{Z} }%Integers
\NewDocumentCommand \Q {} { \mathbb{Q} }%Rational Numbers
\NewDocumentCommand \R {} { \mathbb{R} }%Real Numbers
\NewDocumentCommand \C {} { \mathbb{C} }%Complex Numbers
\NewDocumentCommand \F {} { \mathbb{F} }%Field
\NewDocumentCommand \scrO {} {	\mathscr{O}	}%Dedekind doamin
%Notations for maps
\DeclareMathOperator{\id}{id} % identity
\DeclareMathOperator{\pr}{pr} % projection
\DeclareMathOperator{\pt}{pt} % point
\DeclareMathOperator{\res}{res} % restriction
%PairedDelimiters
\DeclarePairedDelimiterX\abs[1]\lvert\rvert
  { \ifblank{#1}{\:\cdot\:}{#1} }
%abstract value function
\DeclarePairedDelimiterX\norm[1]\lVert\rVert
  { \ifblank{#1}{\:\cdot\:}{#1} }
%the norm function
\DeclarePairedDelimiterX\ceil[1]\lceil\rceil
  { \ifblank{#1}{\:\cdot\:}{#1} }
%ceil function
\DeclarePairedDelimiterX\floor[1]\lfloor\rfloor
  { \ifblank{#1}{\:\cdot\:}{#1} }
%floor function
\DeclarePairedDelimiterX\pairing[2]\langle\rangle
  { \ifblank{#1}{\:\cdot\:}{#1}, \ifblank{#2}{\:\cdot\:}{#2} }
\DeclarePairedDelimiterX\inner[2]\lparen\rparen
  { \ifblank{#1}{\:\cdot\:}{#1}, \ifblank{#2}{\:\cdot\:}{#2} }
%Inner product
\providecommand\given{}
\newcommand\SetSymbol[1][]
  { \nonscript\:#1\vert\allowbreak\nonscript\:\mathopen{} }
\DeclarePairedDelimiterX\Set[1]\{\}
  { \renewcommand\given{\SetSymbol[\delimsize]}#1 }
%a set
%
%Miscellaneous
\NewDocumentCommand \vect { m } { \mathbf{#1} }
%Vector
\RenewDocumentCommand \le {} { \leqslant }
\RenewDocumentCommand \ge {} { \geqslant }
%Change the inequality symbols
\NewDocumentCommand \tforall {} {\quad\text{for all}\quad}
%------------------------
% Additional math symbols, used for the Math 110 course
\DeclareMathOperator*\GCD{GCD}
\DeclareMathOperator*\LCM{LCM}
\usepackage{derivative}% Support typing calculus notations.
\usepackage{polydiv}%Need to manually add the .sty file.
\PolyOptions{variable={{\color{teal}{T}}},base=7,coef=\overline}
%------------------------
\allowdisplaybreaks
%Allow a long equation to display in more than one page.

%The following change the default layout of title
\makeatletter
\def\@maketitle{%
	\newpage
	\null
	\vskip 2em%
	\begin{center}%
		\let \footnote \thanks
		\sffamily 
		{\LARGE \@title \par}%
		\vskip 1.5em%
		{\large \@subtitle \par}%
		\vskip 1em%
		{\large \@date}%
	\end{center}%
	\par
	\vskip 1.5em%
}	
\global\let\@subtitle\@empty
\DeclareRobustCommand*{\subtitle}[1]{\gdef\@subtitle{#1}}
\makeatother
%Leave this change alone if you don't know what it means

%------------------------
%Information of the file
%------------------------
\title{Arithmetic of Polynomials}
%The title of the file
\author{Xu Gao}
%The author
%The subtitle
\date{} 
%The date, if commented, the value will be \today
 
%------------------------
%Document starts here
%------------------------
\begin{document}
\maketitle

\section{Define and display polynomials}
\begin{itemize}
	\item 
Define polynomials by claiming their coefficients:
\begin{verbatim}
	\PolySet{f}{1,0,4,0,2}
	\PolySet{g}{1,6,3}
\end{verbatim}
\PolySet{f}{1,0,4,0,2}
\PolySet{g}{1,6,3}

\item 
Display Polynomials
\begin{verbatim}
	\[\PolyPrint{f}\qquad\PolyPrint{g}\]
\end{verbatim}

\[\PolyPrint{f}\qquad\PolyPrint{g}\]

\item 
\begingroup
Change the variable (globally)
\begin{verbatim}
	\PolyOptions{variable={x}}
	\[\PolyPrint{f}\qquad\PolyPrint{g}\]
\end{verbatim}
\PolyOptions{variable={x}}
\[\PolyPrint{f}\qquad\PolyPrint{g}\]
\endgroup

Or put the option into argument (so change the variable locally),
\begin{verbatim}
	\[\PolyPrint[variable={x}]{f}\qquad\PolyPrint[variable={x}]{g}\]
\end{verbatim}
\[\PolyPrint[variable={x}]{f}\qquad\PolyPrint[variable={x}]{g}\]

\item 
One can also change the modulus:
\begin{verbatim}
	\[\PolyPrint[base=5]{f}\qquad\PolyPrint[base=5]{g}\]
	\[\PolyPrint[base=4]{f}\qquad\PolyPrint[base=3]{g}\]
\end{verbatim}
\[\PolyPrint[base=5]{f}\qquad\PolyPrint[base=5]{g}\]
\[\PolyPrint[base=4]{f}\qquad\PolyPrint[base=3]{g}\]

\item 
Get rid of the bar:
\begin{verbatim}
	\[\PolyPrint[coef=]{f}\qquad\PolyPrint[coef=]{g}\]
\end{verbatim}
\[\PolyPrint[coef=]{f}\qquad\PolyPrint[coef=]{g}\]

\end{itemize}








\section{Arithmetic of polynomials}
We keep use polynomials $f$ and $g$ from previous section (with the modulus $7$).
\PolySet{f}{1,0,4,0,2}
\PolySet{g}{1,6,3}
\begin{itemize}
	\item 
	Addition:
	\begin{verbatim}
		\[\PolyPrint{f}+\PolyPrint{g}=\PolyAdd{f}{g}\]
	\end{verbatim}
	\[\PolyPrint{f}+\PolyPrint{g}=\PolyAdd{f}{g}\]
	\item 
	Multiplication
	\begin{verbatim}
		\[(\PolyPrint{f})(\PolyPrint{g})=\PolyMult{f}{g}\]
	\end{verbatim}
	\[(\PolyPrint{f})(\PolyPrint{g})=\PolyMult{f}{g}\]
	\item 
	Long division
	\begin{verbatim}
		\[\PolyLongDiv{f}{g}\]
	\end{verbatim}
	\[\PolyLongDiv{f}{g}\]
	\item 
	Short division
	\begin{verbatim}
		\[\PolyShortDiv{f}{g}\]
	\end{verbatim}
	\[\PolyShortDiv{f}{g}\]
	\item 
	(Euclidean) division algorithm
	\begin{verbatim}
		\[\PolyEuclid{f}{g}\]
	\end{verbatim}
	\[\PolyEuclid{f}{g}\]
\end{itemize}

\section*{Install}
You need to manually out the ``polydiv.sty'' into your working folder in order to use above.

\clearpage
\section*{Practices}
Try to practice yourself as follows: 
\begin{enumerate}
	\item 
	Choose a modulus and start with any two polynomials $f$ and $g$. 
%-------------Change what below---------------
	\PolyOptions{base=5}
	\PolySet{f}{1,0,4,0,2}
	\PolySet{g}{1,6,3}
%-------------Change what above---------------
	\item 
	Try to do the long division or Euclidean algorithm by yourself. Then verify your answer by running the corresponding code.
%--Uncomment what below to verify your answer----
	% \[\PolyLongDiv{f}{g}\]
	% \[\PolyEuclid{f}{g}\]
%--Uncomment what above to verify your answer----
\end{enumerate}




\end{document}